% !TEX TS-program = pdflatex

%%%%%%%%%% NO MODIFICAR  %%%%%%
\documentclass[journal]{IEEEtran}
\IEEEoverridecommandlockouts
\input{Configuracion}
%%%%%%%%%%%%%%%%%%%%%%%%%%%%%%%%


%%%%% INICIO DEL DOCUMENTO %%%%%
%      AREA CONFIGURABLE       %
%%%%%%%%%%%%%%%%%%%%%%%%%%%%%%%%
\begin{document}

%Titulo
\title{\textcolor{black}{Report on 10.2-10.8}}      

%Datos
\author{Shenyao Jin} 


\maketitle


\section{Code Snapshot}
\begin{verbatim}
# Homework 1, CV
# Shenyao Jin, shenyaojin@mines.edu
#%% Import libs
import numpy as np
import matplotlib.pyplot as plt
from skimage.io import imread
#%% Define data path
data_path = "./data/IMG_5167.jpeg"

#%% Load image
image = imread(data_path)
rotated_image = np.rot90(image, k=3) # Need to rotate the image
#%% Show image
plt.figure()
plt.imshow(rotated_image)
plt.axis("off")
plt.show()

#%% Change the dtype to float32 and convert into grayscale
image_float = rotated_image.astype(np.float32)
image_gray = np.dot(image_float, [0.299, 0.587, 0.114])

#%% Show grayscale image
plt.figure()
plt.imshow(image_gray, cmap='gray')
plt.axis("off")
plt.show()

#%% Print the top left (3*5) pixels of the grayscale image
print(image_gray[:3,:5])

#%% Print index=(1,2) of image_gray. We assume the index start from 1.
print(image_gray[0,1])
\end{verbatim}

\section{Grayscale Image Snapshot}
\begin{figure}[h]
    \centering
    \includegraphics[width=0.8\columnwidth]{../data/gray_scale_figure.png}
    \caption{Grayscale Image}
\end{figure}

\section{Pixel Values}

\subsection{Top Left 3x5 Matrix}
\begin{verbatim}
[[179.789 180.789 177.789 172.789 170.789]
 [178.789 178.789 176.789 174.789 171.789]
 [179.789 178.789 177.789 177.789 177.789]]
\end{verbatim}

\subsection{Pixel at (1,2)}
The value of the pixel at index (1,2) is 180.789.


%%%%%%%%%%%%%%%%%%%%%%%%%%%%%%%
%%%%%%%%%%%%%%%%%%%%%%%%%%%%%%%
\end{document}






